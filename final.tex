\pagestyle{empty}
\cleardoublepage
\pagestyle{fancy}
\chapter{Direções Futuras}\label{cap5}

Neste capítulo vamos indicar alguns temas possíveis para serem abordados na dissertação.
Podemos aproveitar os resultados atuais da Complexidade Descritiva com lógicas convencionais para guiar na tentativa de mostrar resultados semelhantes para lógicas probabilísticas. Resolvemos estudar Lógicas Probabilísticas de Primeira Ordem pois já sabemos que $FO = LH$ \cite{immerman99}, $\exists SO = NP$ e que as classes de complexidade probabilísticas $RP$, $ZPP$ e $BPP$ estão entre as classes $P$ e $NP$. Com isso, buscamos mostrar a relação entre lógicas probabilísticas e as classes de complexidade computacional probabilísticas mencionadas. Outro tipo de lógica probabilística que existe adiciona probabilidade no valor verdade da fórmulas. Poderíamos investigar essas lógicas e buscar resultados de complexidade descritiva para elas. Outra direção seria adicionar probabilidade em outras lógicas conhecidas que já têm resultados em Complexidade Descritiva afim de obter resultados que relacionem com as classes de complexidade probabilísticas. Por exemplo, sabemos que a lógica de primeira ordem com operador de menor ponto fico $FO(FLP)$ e a lógica de Horn de segunda ordem caracterizam a classe de complexidade $P$ \cite{immerman99}. Poderíamos investigar se já existem versões probabilísticas dessas lógicas ou até mesmo adicionar probabilidade com a finalidade de capturar alguma classe de complexidade probabilística. Também existem resultados de complexidade descritiva para algumas lógicas modais \cite{Cibele2009} e o mesmo poderia ser feito com suas versões probabilísticas \cite{BrazdilFKK08, Heifetz1998}.

\section{Cronograma}
Nesta seção temos o cronograma de estudo e trabalho para conclusão da dissertação de mestrado.

\begin{itemize}
 \item Março: Tentar provar os resultados e estudar outras abordagens.
 \item Abril: Participação no Unilog e apresentar a pré-proposta na PUC e UFRJ.
 \item Maio: Incorporação de sugestões ao trabalho.
 \item Junho a Agosto: Escrever a proposta com possíveis resultados.
 \item Agosto a Novembro: Escrever artigo com possíveis resultados.
 \item Novembro a Fevereiro: Escrever dissertação com resultados obtidos.
\end{itemize}