% Limpa cabeçalhos.
% (solução para lidar com a númeração das páginas pré-textuais).
\pagestyle{empty}

%% Capa
\begin{titlepage}

% Se quiser uma figura de fundo na capa ative o pacote wallpaper
% e descomente a linha abaixo.
% \ThisCenterWallPaper{0.8}{nomedafigura}

\begin{center}
{\LARGE Daniel Nascimento Teixeira}
\par
\vspace{200pt}
{\Huge Técnicas de Particionamento para Malhas}
\par
\vfill
\textbf{{\large Fortaleza}\\
{\large \the\year}}
\end{center}
\end{titlepage}

% Faz com que a página seguinte sempre seja ímpar (insere pg em branco)
\cleardoublepage

% Numeração em elementos pré-textuais é opcional (ativada por padrão).
% Para desativá-la comente a linha abaixo.
\pagestyle{fancy}

% Números das páginas em algarismos romanos
\pagenumbering{roman}

%% Página de Rosto

% Numeração não deve aparecer na página de rosto.
\thispagestyle{empty}

\begin{center}
{\LARGE Daniel Nascimento Teixeira}
\par
\vspace{200pt}
{\Huge Técnicas de Particionamento para Malhas}
\end{center}
\par
\vspace{90pt}
\hspace*{175pt}\parbox{7.6cm}{{\large Dissertação apresentada ao Departamento de Computação da Universidade Federal do Ceará, para a obtenção de Título de Mestre}}

\par
\vspace{1em}
\hspace*{175pt}\parbox{7.6cm}{{\large Orientador: Joaquim Bento}}

\par
\vfill
\begin{center}
\textbf{{\large Fortaleza}\\
{\large \the\year}}
\end{center}

\newpage

% Ficha Catalográfica
\hspace{8em}\fbox{\begin{minipage}{10cm}
Aluno, Daniel Nascimento Teixeira.

\hspace{2em} Técnicas de Particionamento para Malhas

\hspace{2em}\pageref{LastPage} páginas

\hspace{2em}Pré-proposta (Mestrado) - Universidade Federal do Ceará. Departamento de Computação.

\begin{enumerate}
\item Malhas
\item Subdivisão de domínio
\item Paralelismo
\end{enumerate}
I. Universidade Federal do Ceará. Departamento de Computação.

\end{minipage}}
\par
\vspace{2em}
\begin{center}
{\LARGE\textbf{Comissão Julgadora:}}

\par
\vspace{10em}
\begin{tabular*}{\textwidth}{@{\extracolsep{\fill}}l l}
\rule{16em}{1px} 	& \rule{16em}{1px} \\
Prof. Dr. 		& Prof.ª Dr. \\
Creto Vidal			& Emanuele
\end{tabular*}

\par
\vspace{10em}

\parbox{16em}{\rule{16em}{1px} \\
Prof. Dr. \\
Joaquim Bento}
\end{center}

\newpage

% Epígrafe
\vspace*{0.4\textheight}
\noindent{\LARGE\textbf{Epígrafe}}
% Tudo que você escreve no verbatim é renderizado literalmente (comandos não são interpretados e os espaços são respeitados)
\begin{verbatim}
Aprendi através da experiência amarga a suprema lição: 
controlar minha ira e torná-la como o calor que é convertido em energia. 
Nossa ira controlada pode ser convertida numa força capaz de mover o mundo.
\end{verbatim}
\begin{flushright}
Mahatma Gandhi
\end{flushright}

\newpage

% Agradecimentos

% Espaçamento duplo
\doublespacing

\noindent{\LARGE\textbf{Agradecimentos}}

Agradeço ao meu orientador, ao meu co-orientador, aos meus colaboradores, aos técnicos, à seção administrativa, à fundação que liberou verba para minhas pesquisas, aos meus amigos e à minha família.

\newpage

%\vspace*{10pt}
% Abstract
%\begin{center}
%  \emph{\begin{large}Resumo\end{large}}\label{resumo}
%\vspace{2pt}
%\end{center}
% Pode parecer estranho, mas colocar uma frase por linha ajuda a organizar e reescrever o texto quando necessário.
% Além disso, ajuda se você estiver comparando versões diferentes do mesmo texto.
% Para separar parágrafos utilize uma linha em branco.
%\noindent
%Este trabalho descreve uma técnica de geração de malhas bidimensionais triangulares usando computadores paralelos com memória distribuída, baseado em um modelo de paralelismo mestre/escravos. Esta técnica utiliza uma \textit{quadtree} grosseira para decompor o domínio e um processo de avanço de fronteira serial para gerar a malha em cada subdomínio simultaneamente. Uma \textit{quadtree} fina é empregada neste trabalho para ajudar a estimar a carga de processamento associado a cada subdomínio. O nível de refinamento da \textit{quadtree} fina é usado para orientar a criação das arestas que são definidas pela discretização das células internas da \textit{quadtree} dos subdomínios. É mostrado que a técnica de estimativa de carga produz resultados que representam com precisão o número de elementos a serem gerados em cada subdomínio, levando a previsão de tempo de execução adequada e com um algoritmo bem balanceado. Quando comparado com o seu homólogo serial, a técnica em paralelo é muito mais rápida e muitas vezes apresenta um desempenho superlinear. As malhas geradas com a técnica em paralelo têm a mesma qualidade como as geradas serialmente, dentro de limites aceitáveis​​. 
%\par
%\vspace{1em}
%\noindent\textbf{Palavras-chave:} Malha triangular, Subdivisão de domínio, Quadtree
%\newpage

% Criei a página do abstract na mão, por isso tem bem mais comandos do que o resumo acima, apesar de serem idênticas.
%\vspace*{10pt}
% Abstract
%\begin{center}
%  \emph{\begin{large}Abstract\end{large}}\label{abstract}
%\vspace{2pt}
%\end{center}

% Selecionar a linguagem acerta os padrões de hifenação diferentes entre inglês e português.
%\selectlanguage{english}
%\noindent
%This work describes a technique for generating two-dimensional triangular meshes using parallel computers with distributed memory, based on a master/slaves parallelism model. This technique uses a coarse quadtree to decompose the domain and a serial advancing front procedure to generate the mesh in each subdomain concurrently. A finer quadtree is employed in this work to help estimate the processing load associated with each subdomain. The level of refinement of this finer quadtree is used to guide the creation of the edges that define the quadtree's inter-cell discretization of the subdomains. It is shown that the load estimation technique produces results that accurately represent the number of elements to be generated in each subdomain, leading to proper runtime prediction and to a well-balanced algorithm. When compared with its serial counterpart, the parallel technique is much faster and often presents a super-linear performance. The meshes generated with the parallel technique have the same quality as those generated serially, within acceptable limits.
%\par
%\vspace{1em}
%\noindent\textbf{Keywords:} Triangular mesh, Domain subdivision, Quadtree

% Voltando ao português...
%\selectlanguage{brazilian}

%\newpage

% Desabilitar protrusão para listas e índice
\microtypesetup{protrusion=false}

% Lista de figuras
\listoffigures

% Lista de tabelas
%\listoftables

% Abreviações
% Para imprimir as abreviações siga as instruções em 
% http://code.google.com/p/mestre-em-latex/wiki/ListaDeAbreviaturas
%\printnomenclature

% Índice
\tableofcontents

% Re-habilita protrusão novamente
\microtypesetup{protrusion=true}
